%%%%%%%%%%%%%%%%%%%%%%%%%%%%%%%%%%%%%%%%%%%%%%%%%%%%%%%%%%%%%%%%%%%%%
% LaTeX Template: Project Titlepage Modified (v 0.1) by rcx
%
% Original Source: http://www.howtotex.com
% Date: February 2014
% 
% This is a title page template which be used for articles & reports.
% 
% This is the modified version of the original Latex template from
% aforementioned website.
% 
%%%%%%%%%%%%%%%%%%%%%%%%%%%%%%%%%%%%%%%%%%%%%%%%%%%%%%%%%%%%%%%%%%%%%%

\documentclass[12pt]{llncs} %{article}
\usepackage[a4paper]{geometry}
\usepackage[myheadings]{fullpage}
\usepackage{fancyhdr}
\usepackage{lastpage}
\usepackage{graphicx, wrapfig, subcaption, setspace, booktabs}
\usepackage[T1]{fontenc}
\usepackage[font=small, labelfont=bf]{caption}
\usepackage{fourier}
\usepackage[protrusion=true, expansion=true]{microtype}
\usepackage[english]{babel}
\usepackage{sectsty}
\usepackage{url, lipsum}
\usepackage[utf8]{inputenc}
\usepackage{float}
\restylefloat{table}


\newcommand{\HRule}[1]{\rule{\linewidth}{#1}}


\begin{document}

\title{Letter dataset: Results of Text mining project 2020}
\date{}

\author{Your name and UCO}
\institute{PA164 and KD Lab FI MU Brno}
\maketitle

\begin{abstract}
This work addresses the problem of ...

\end{abstract}

\section{Data set and task description}

\subsection{Data set}
link to the data set, number of instances, features, missing values etc , data preparation (e.g. transformation into form)

\subsection{Data set reduction}

\subsubsection{Feature reduction}
introduce the learning curve, explain what number of feature you will use and why

\subsubsection{Other reduction}
voluntary; only in the case that you need eg. limit the number of instances

\section{Description of the method used}
\subsection{General}
what classifiers (CLF) and what outlier detection (OD) methods you used.
Do not describe them.

\subsection{Document-term matrix}
what representation (pre-processing, PP) you used (very likely binary, TF, TF/IDF)

\section{Results}
\subsection{Overview}
general description of your results

\subsection{Comparison of different combinations of (PP+OD+CLF)}

including graphs and statistical tests.

Which combination was the best in term of accuracy.

Is that combination PP+OD+CLF significantly better that the others?
%https://is.muni.cz/auth/el/fi/podzim2020/PA164/projekt/Statistical_Comparison_of_Classifiers_over_Multiple_Data_Sets.pdf}

\subsection{Discussion} 
including discussion of the outliers that you found interesting

\section{Conclusion}

\section{References}
only if you employed something uncommon.
\end{document}